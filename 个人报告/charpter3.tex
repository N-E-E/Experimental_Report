\documentclass[supercite]{Experimental_Report}

\usepackage{algorithm, multirow}
\usepackage{algpseudocode}
\usepackage{amsmath}
\usepackage{amsthm}
\usepackage{framed}
\usepackage{mathtools}
\usepackage{subcaption}
\usepackage{xltxtra} %提供了针对XeTeX的改进并且加入了XeTeX的LOGO, 自动调用xunicode
%宏包(提供Unicode字符宏)
\usepackage{bm}
\usepackage{tikz}
\usepackage{tikzscale}
\usepackage{pgfplots}
\usepackage{listings}
\usepackage{xcolor}
\usepackage{fontspec}

%%%%%%settings%%%%%%%%%
\pgfplotsset{compat=1.16}
\setmonofont{Consolas}

\definecolor{mygreen}{rgb}{0,0.6,0}
\definecolor{mygray}{rgb}{0.5,0.5,0.5}
\definecolor{mymauve}{rgb}{0.58,0,0.82}
\lstset{ %
	backgroundcolor=\color{white},   % choose the background color
	basicstyle=\footnotesize\ttfamily,        % size of fonts used for the code
	columns=fullflexible,
	breaklines=true,                 % automatic line breaking only atwhitespace
	captionpos=b,                    % sets the caption-position to bottom
	tabsize=4,
	commentstyle=\color{mygreen},    % comment style
	escapeinside={\%*}{*)},          % if you want to add LaTeX within your code
	keywordstyle=\color{blue},       % keyword style
	stringstyle=\color{mymauve}\ttfamily,     % string literal style
	frame=single,
	rulesepcolor=\color{red!20!green!20!blue!20},
	% identifierstyle=\color{red},
	language=c++,
}

\begin{document}
	\section{附录C 基于二叉链表存储结构实现二叉树的源程序}
	\begin{center}
		\textbf{/* Binary Tree On Binary LinkList Structure */}
	\end{center}
	main.cpp
	\begin{lstlisting}
		#include "def.h"
		#include "SingleTree.h"
		#include "Manager.h"
		
		int main() {
			int state, op, index, e;
			char filename[30];
			BiTree tmp = nullptr;
			TElemType value;
			Manager M;
			Menu();
			cout << "enter your choice:" << endl;
			cin >> op;
			while (op){
				switch (op) {
					case 1:  // 添加成员
					state = M.AddMember();
					if (state == OK){
						cout << "add successfully. The preorder series is: ";
						traverse(M.member[M.length-1].T);
						cout << endl;
						cout << "the structure now is:" << endl;
						M.DispStructure();
						cout << '\n';
					}
					else{
						cout << "false order or duplicated key!" << endl;
					}
					break;
					case 2:  // 删除成员
					state = M.DelMember();
					if (state == OK) {
						cout << "delete successfully, the structure now is:" << 
						endl;
						M.DispStructure();
					}
					else cout << "empty tree!" << endl;
					break;
					case 3:  // 显示树形目录
					M.DispStructure();
					break;
					case 4:  // 初始化
					index = M.GetCommand1();
					if (index != -1){
						if (IsEmpty(M.member[index].T) == false){
							cout << "existed tree!" << endl;
						}
						else{
							TElemType definition[100];
							GetData(definition);
							state = CreateBiTree(M.member[index].T, definition);
							if (state == OK) {
								cout << "create successfully! Now the structure 
								is: " << endl;
								M.DispStructure();
							}
							else cout << "error!" << endl;
						}
					}
					break;
					case 5:  // 清空二叉树
					if ((index = M.GetCommand1()) != -1){
						if (IsEmpty(M.member[index].T) == true){
							cout << "empty tree!" << endl;
						}
						else{
							state = ClearBiTree(M.member[index].T);
							if (state == OK) {
								cout << "clear successfully! Now the structure 
								is: " << endl;
								M.DispStructure();
							}
							else cout << "error!" << endl;
						}
					}
					break;
					case 6:  // 获取树深
					if ((index = M.GetCommand1()) != -1){
						if (IsEmpty(M.member[index].T) == true){
							cout << "empty tree!" << endl;
						}
						else{
							cout << "The depth is: " << 
							GetDepth(M.member[index].T) <<endl;
						}
					}
					break;
					case 7:  // 查找节点
					if ((index = M.GetCommand1()) != -1){
						if (IsEmpty(M.member[index].T) == true){
							cout << "empty tree!" << endl;
						}
						else{
							cout << "enter the key:" << endl;
							cin >> e;
							tmp = find(M.member[index].T, e);
							if (tmp){
								cout << "result: " << tmp->data.key << "," << 
								tmp->data.others << endl;
							}
							else cout << "no results." << endl;
						}
					}
					break;
					case 8:  // 节点赋值
					if ((index = M.GetCommand1()) != -1){
						if (IsEmpty(M.member[index].T) == true){
							cout << "empty tree!" << endl;
						}
						else{
							cout << "enter the key and new key-value pair:" << 
							endl;
							int key;
							cin >> key >> value.key >> value.others;
							state = Assign(M.member[index].T, key, value);
							if (state == -1) cout << "duplicated key!" << endl;
							else if (state == ERROR) cout << "can't find the 
							corresponding node!" << endl;
							else {
								cout << "assign successfully! The preorder is: 
								" << endl;
								PreOrder(M.member[index].T);
							}
						}
					}
					break;
					case 9:  // 获取兄弟
					if ((index = M.GetCommand1()) != -1){
						if (IsEmpty(M.member[index].T) == true){
							cout << "empty tree!" << endl;
						}
						else{
							cout << "enter the key:" << endl;
							cin >> e;
							tmp = GetSibling(M.member[index].T, e);
							if (!tmp) cout << "can't find sibling!" << endl;
							else{
								cout << "Sibling is " << tmp->data.key << "," 
								<< tmp->data.others << endl;
							}
						}
					}
					break;
					case 10:  // 插入节点
					if ((index = M.GetCommand1()) != -1){
						if (IsEmpty(M.member[index].T) == true){
							cout << "empty tree!" << endl;
						}
						else{
							cout << "enter the key:" << endl;
							cin >> e;
							cout << "enter the new node info: " << endl;
							cin >> value.key >> value.others;
							cout << "how to insert(enter -1 or 0 or 1): "  << 
							endl;
							int lr;
							cin >> lr;
							state = InsertNode(M.member[index].T, e, lr, value);
							if (state == -1) cout << "duplicated key!" << endl;
							else if (state == ERROR) cout << "can't find the 
							key!" << endl;
							else{
								cout << "insert over. The preorder traverse is: 
								" << endl;
								PreOrder(M.member[index].T);
							}
						}
					}
					break;
					case 11:  // 删除节点
					if ((index = M.GetCommand1()) != -1){
						if (IsEmpty(M.member[index].T) == true){
							cout << "empty tree!" << endl;
						}
						else{
							cout << "enter the key:" << endl;
							cin >> e;
							state = DeleteNode(M.member[index].T, e);
							if (state == ERROR) cout << "can't find the node!" 
							<< endl;
							else{
								cout << "delete over. The preorder is: " << 
								endl;
								PreOrder(M.member[index].T);
							}
						}
					}
					break;
					case 12:  // 前序遍历
					if ((index = M.GetCommand1()) != -1){
						if (IsEmpty(M.member[index].T) == true){
							cout << "empty tree!" << endl;
						}
						else{
							PreOrder(M.member[index].T);
							cout << endl;
						}
					}
					break;
					case 13:  // 中序遍历
					if ((index = M.GetCommand1()) != -1){
						if (IsEmpty(M.member[index].T) == true){
							cout << "empty tree!" << endl;
						}
						else{
							InOrder(M.member[index].T);
							cout << endl;
						}
					}
					break;
					case 14:  // 后序遍历
					if ((index = M.GetCommand1()) != -1){
						if (IsEmpty(M.member[index].T) == true){
							cout << "empty tree!" << endl;
						}
						else{
							PostOrder(M.member[index].T);
							cout << endl;
						}
					}
					break;
					case 15:  // 层序遍历
					if ((index = M.GetCommand1()) != -1){
						if (IsEmpty(M.member[index].T) == true){
							cout << "empty tree!" << endl;
						}
						else{
							LevelOrder(M.member[index].T);
							cout << endl;
						}
					}
					break;
					case 16:  // 保存文件
					if ((index = M.GetCommand1()) != -1){
						if (IsEmpty(M.member[index].T) == true){
							cout << "empty tree!" << endl;
						}
						else{
							cout << "enter the filename:" << endl;
							cin >> filename;
							state = SaveBiTree(M.member[index].T, filename);
							if (state == ERROR) cout << "IO Error!" << endl;
							else cout << "save successfully!" << endl;
						}
					}
					break;
					case 17:  // 载入文件
					if ((index = M.GetCommand1()) != -1){
						if (IsEmpty(M.member[index].T) == false){
							cout << "tree existed!" << endl;
						}
						else{
							cout << "enter the filename:" << endl;
							cin >> filename;
							state = LoadBiTree(M.member[index].T, filename);
							if (state == ERROR) cout << "IO Error!" << endl;
							else cout << "load successfully!" << endl;
						}
					}
					break;
					case 18:  // 最大路径和
					if ((index = M.GetCommand1()) != -1){
						cout << "max sum is: " << MaxPathSum(M.member[index].T) 
						<< endl;
					}
					break;
					case 19:  // lca问题
					if ((index = M.GetCommand1()) != -1){
						cout << "enter 2 child" << endl;
						int e1, e2;
						cin >> e1 >> e2;
						tmp = LCA(M.member[index].T, e1, e2);
						cout << "the ancestor is: " << tmp->data.key << "," << 
						tmp->data.others << endl;
					}
					break;
					case 20:  // 翻转二叉树
					if ((index = M.GetCommand1()) != -1){
						InvertTree(M.member[index].T);
						cout << "Invert over, the preorder is:" << endl;
						PreOrder(M.member[index].T);
					}
					break;
					default:
					cout << "wrong command!" << endl;
					break;
					case 21:
					if ((index = M.GetCommand1()) != -1){
						auto judge = IsEmpty(M.member[index].T);
						if (judge == true) cout << "empty!" << endl;
						else cout << "not empty!" << endl;
					}
					break;
				}
				Menu();
				cout << "enter your command:" << endl;
				cin >> op;
			}
			cout << "bye!" << endl;
			return 0;
		}
	\end{lstlisting}
	def.h
	\begin{lstlisting}
		#pragma once
		#include "cstdio"
		#include "cstdlib"
		#include <string>
		#include <iostream>
		#include <queue>
		#include <map>
		using namespace std;
		
		#define TRUE 1
		#define FALSE 0
		#define OK 1
		#define ERROR 0
		#define INFEASIBLE -1
		#define OVERFLOW -2
		#define MaxSize 50
		
		typedef int status;
		typedef int KeyType;
		typedef struct TreeElem{
			KeyType  key;
			char others[30];
		} TElemType; //二叉树结点类型定义
		
		
		typedef struct BiTNode{  //二叉链表结点的定义
			TElemType  data;
			struct BiTNode *lchild,*rchild;
			BiTNode() : lchild(nullptr), rchild(nullptr){}
		} BiTNode, *BiTree;
		
		typedef struct TreeUnit{
			string name;
			BiTree  T;
			TreeUnit() : name("unnamed"), T(nullptr){}
		} TU;
	\end{lstlisting}
	SingleTree.h	
	\begin{lstlisting}
		#pragma once
		#include "def.h"
		
		//展示菜单
		void Menu();
		
		//判断是否是空树
		status IsEmpty(BiTree T);
		
		//获取先序初始化数组
		void GetData(TElemType definition[]);
		
		// 由带空节点的前序创建二叉树
		status CreateBiTree(BiTree &T,TElemType definition[]);
		BiTree build(BiTree &cur, TElemType definition[], int &cnt);  // 辅助函数
		void traverse(BiTree T);  // 先序遍历判断创建是否成功
		
		//清空二叉树
		status ClearBiTree(BiTree &T);
		void clear(BiTree T);  // 辅助函数
		
		//求二叉树深度
		int GetDepth(BiTree T);
		
		//查找给定关键词的节点
		BiTree find(BiTree cur, KeyType e);
		
		//节点赋值,要求保证关键词唯一性
		status Assign(BiTree &T,KeyType e,TElemType value);
		int traverse(BiTree T, KeyType e, TElemType value);
		
		//获取兄弟节点
		BiTNode* GetSibling(BiTree T,KeyType e);
		
		//插入节点
		status InsertNode(BiTree &T,KeyType e,int LR,TElemType c);
		int traverse1(BiTree T, KeyType e, TElemType value);  // 辅助判断键唯一性
		
		//删除节点
		status DeleteNode(BiTree &T,KeyType e);
		BiTree delete_(BiTree cur, KeyType e);  // 删除辅助函数
		int traverse2(BiTree T, KeyType e);  // 判断是否有目标
		
		//前序遍历
		void PreOrder(BiTree T);
		
		//中序遍历
		void InOrder(BiTree T);
		
		//后序遍历
		void PostOrder(BiTree T);
		
		//层序遍历
		void LevelOrder(BiTree T);
		
		//保存到文件
		status SaveBiTree(BiTree T, char FileName[]);
		void save_traverse(BiTree T, FILE* fp);  // 保存文件辅助函数
		
		//从文件读取
		status LoadBiTree(BiTree &T,  char FileName[]);
		BiTree read(BiTree T, FILE* fp);  //文件读取辅助函数
		
		//最大路径和
		int MaxPathSum(BiTree T);
		
		//最近公共祖先
		BiTree LCA(BiTree T, KeyType e1, KeyType e2);
		int FindChild(BiTree T, KeyType e);
		
		//翻转二叉树
		void InvertTree(BiTree &T);
	\end{lstlisting}
	SingleTree.cpp
	\begin{lstlisting}
		#include "SingleTree.h"
		
		void Menu(){
			cout << "                  Menu" << endl;
			cout << "--------------------------------------------" << endl;
			cout << "    1. NewTree                 2. DelTree" << endl;
			cout << "    3. DispStructure           " << endl;
			cout << endl;
			cout << "    4. CreateBiTree            5. ClearBiTree" << endl;
			cout << "    6. GetDepth                7. FindNode" << endl;
			cout << "    8. TreeNodeAssign          9. GetSibling" << endl;
			cout << "    10. InsertNode             11. DeleteNode" << endl;
			cout << "    12. PreOrder               13. InOrder" << endl;
			cout << "    14. PostOrder              15. LevelOrder" << endl;
			cout << "    16. Save                   17. Load" << endl;
			cout << "    18. MaxPathSum             19. LCA" << endl;
			cout << "    20. InvertTree             21. IsEmpty" << endl;
			cout << "    0. quit" << endl;
			cout << "--------------------------------------------" << endl;
		}
		
		
		status IsEmpty(BiTree T){
			if (!T) return true;
			else return false;
		}
		
		
		void GetData(TElemType definition[]){
			cout << "enter the key-value pair(preorder) end with key \'-1\':" 
			<< endl;
			int key, i = 0;
			string value;
			cin >> key;
			while (key != -1){
				definition[i].key = key;
				scanf("%s", definition[i].others);
				i++;
				cin >> key;
			}
			definition[i].key = -1;
			scanf("%s", definition[i].others);
			cout << "read successfully!" << endl;
		}
		
		
		status CreateBiTree(BiTree &T,TElemType definition[])
		/*根据带空枝的二叉树先根遍历序列definition构造一棵二叉树,将根节点指针赋值给T
		并返回OK,
		如果有相同的关键字,返回ERROR.*/
		{
			int hash[200] = {0};
			int i = 0;
			while (definition[i].key != -1){
				if (hash[definition[i].key] == 1 && definition[i].key != 0){
					return ERROR;
				}
				hash[definition[i].key] = 1;
				i++;
			}
			int cnt = 0;
			T = build(T, definition, cnt);
			return OK;
		}
		
		BiTree build(BiTree &cur, TElemType definition[], int& cnt){
			if (definition[cnt].key == 0){
				cnt++;
				return nullptr;
			}
			cur = (BiTree)malloc(sizeof(struct BiTNode));
			cur->data = definition[cnt++];
			cur->lchild = build(cur->lchild, definition, cnt);
			cur->rchild = build(cur->rchild, definition, cnt);
			return cur;
		}
		
		void traverse(BiTree T){
			if (T == nullptr) return;
			printf("%d,%s   ", T->data.key, T->data.others);
			traverse(T->lchild);
			traverse(T->rchild);
		}
		
		status ClearBiTree(BiTree &T)
		//将二叉树设置成空,并删除所有结点,释放结点空间
		{
			// 请在这里补充代码,完成本关任务
			/********** Begin *********/
			if (T == nullptr) return ERROR;
			clear(T);
			T = nullptr;
			return OK;
			
			/********** End **********/
		}
		
		void clear(BiTree T){
			if (T == nullptr) return;
			clear(T->lchild);
			clear(T->rchild);
			free(T);
		}
		
		
		int GetDepth(BiTree T){
			if (T == nullptr) return 0;
			int l = GetDepth(T->lchild) + 1;
			int r = GetDepth(T->rchild) + 1;
			if (l >= r) return l;
			else return r;
		}
		
		
		BiTree find(BiTree cur, KeyType e){
			if (cur == nullptr) return nullptr;
			if (cur->data.key == e) return cur;
			BiTree ansl = find(cur->lchild, e);
			BiTree ansr = find(cur->rchild, e);
			if (ansl) return ansl;
			if (ansr) return ansr;
			return nullptr;
		}
		
		
		status Assign(BiTree &T,KeyType e,TElemType value){
			if (traverse(T, e, value) == 0) return -1;  // -1表示键重复
			BiTree ans = find(T, e);
			if (!ans) return ERROR;
			else{
				ans->data = value;
				return OK;
			}
		}
		
		int traverse(BiTree T, KeyType e, TElemType value){
			if (T == nullptr) return 1;
			if (T->data.key != e && T->data.key == value.key) return 0;
			else return traverse(T->lchild, e, value) && traverse(T->rchild, e, 
			value);
		}
		
		
		BiTNode* GetSibling(BiTree T,KeyType e){
			if (T == nullptr) return nullptr;
			if (!(T->lchild && T->rchild)) return nullptr;
			if (T->lchild->data.key == e) return T->rchild;
			if (T->rchild->data.key == e) return T->lchild;
			BiTree l = GetSibling(T->lchild, e);
			if (l) return l;
			BiTree r = GetSibling(T->rchild, e);
			if (r) return r;
			return nullptr;
		}
		
		
		status InsertNode(BiTree &T,KeyType e,int LR,TElemType c){
			if (traverse1(T, e, c) == 0) return -1;
			BiTree ans = find(T, e);
			if (!ans) return ERROR;
			BiTree new_node = (BiTree)malloc(sizeof(struct BiTNode));
			if (LR == 0){
				new_node->rchild = ans->lchild;
				new_node->lchild = nullptr;
				new_node->data = c;
				ans->lchild = new_node;
			}
			else if (LR == -1){
				new_node->rchild = T;
				new_node->lchild = nullptr;
				new_node->data = c;
				T = new_node;
			}
			else{
				new_node->rchild = ans->rchild;
				new_node->lchild = nullptr;
				new_node->data = c;
				ans->rchild = new_node;
			}
			return OK;
		}
		
		int traverse1(BiTree T, KeyType e, TElemType value){
			if (T == nullptr) return 1;
			if (T->data.key == value.key) return 0;
			else return traverse(T->lchild, e, value) && traverse(T->rchild, e, 
			value);
		}
		
		
		status DeleteNode(BiTree &T,KeyType e){
			if (!traverse2(T, e)) return ERROR;
			T = delete_(T, e);
			return OK;
		}
		
		BiTree delete_(BiTree cur, KeyType e){
			if (!cur) return nullptr;
			if (cur->data.key != e){
				cur->lchild = delete_(cur->lchild, e);
				cur->rchild = delete_(cur->rchild, e);
			}
			else{
				if (!cur->lchild && !cur->rchild){
					free(cur);
					return nullptr;
				}
				else if (!cur->lchild){
					BiTree p = cur->rchild;
					free(cur);
					return p;
				}
				else if (!cur->rchild){
					BiTree p = cur->rchild;
					free(cur);
					return p;
				}
				else{
					BiTree p = cur->lchild;
					BiTree p0 = p;
					while (p0->rchild){
						p0 = p0->rchild;
					}
					p0->rchild = cur->rchild;
					free(cur);
					return p;
				}
			}
			return cur;
		}
		
		int traverse2(BiTree T, KeyType e){
			if (T == nullptr) return 0;
			if (T->data.key == e) return 1;
			return (traverse2(T->lchild, e) || traverse2(T->rchild, e));
		}
		
		
		void PreOrder(BiTree T){
			if (T == nullptr) return;
			cout << T->data.key << "," << T->data.others << " " << endl;
			PreOrder(T->lchild);
			PreOrder(T->rchild);
		}
		
		
		void InOrder(BiTree T){
			if (T == nullptr) return;
			InOrder(T->lchild);
			cout << T->data.key << "," << T->data.others << " " << endl;
			InOrder(T->rchild);
		}
		
		
		void PostOrder(BiTree T){
			if (T == nullptr) return;
			PostOrder(T->lchild);
			PostOrder(T->rchild);
			cout << T->data.key << "," << T->data.others << " " << endl;
		}
		
		
		void LevelOrder(BiTree T){
			queue<BiTree> q;
			BiTree tmp;
			q.push(T);
			while (!q.empty()){
				tmp = q.front();
				q.pop();
				cout << tmp->data.key << "," << tmp->data.others << " " << endl;
				if (tmp->lchild) q.push(tmp->lchild);
				if (tmp->rchild) q.push(tmp->rchild);
			}
		}
		
		
		status SaveBiTree(BiTree T, char FileName[])
		{
			FILE* fp = fopen(FileName, "w");
			if (fp){
				save_traverse(T, fp);
				fclose(fp);
				return OK;
			}
			fclose(fp);
			return ERROR;
		}
		
		void save_traverse(BiTree T, FILE* fp){
			if (!T) {
				fprintf(fp, "%d\n", 0);
				return;
			}
			fprintf(fp, "%d %s\n", T->data.key, T->data.others);
			save_traverse(T->lchild, fp);
			save_traverse(T->rchild, fp);
		}
		
		
		status LoadBiTree(BiTree &T,  char FileName[])
		{
			FILE* fp = fopen(FileName, "r");
			if (fp){
				T = read(T, fp);
				fclose(fp);
				return OK;
			}
			fclose(fp);
			return ERROR;
		}
		
		BiTree read(BiTree T, FILE* fp){
			int key;
			fscanf(fp, "%d", &key);
			if (key == 0){
				return nullptr;
			}
			else{
				T = (BiTree)malloc(sizeof(struct BiTNode));
				T->data.key = key;
				fscanf(fp, "%s", T->data.others);
				T->lchild = read(T->lchild, fp);
				T->rchild = read(T->rchild, fp);
				return T;
			}
		}
		
		
		int MaxPathSum(BiTree T){
			if (T == nullptr) return 0;
			int l = MaxPathSum(T->lchild);
			int r = MaxPathSum(T->rchild);
			return T->data.key + max(l, r);
		}
		
		
		BiTree LCA(BiTree T, KeyType e1, KeyType e2){
			int l1 = FindChild(T->lchild, e1);
			int l2 = FindChild(T->rchild, e1);
			int r1 = FindChild(T->lchild, e2);
			int r2 = FindChild(T->rchild, e2);
			if (l1 && r2 || l2 && r1) return T;
			if (T->data.key == e1 || T->data.key == e2) return T;
			if (l1 && r1) return LCA(T->lchild, e1, e2);
			else return LCA(T->rchild, e1, e2);
		}
		
		int FindChild(BiTree T, KeyType e){
			if (T == nullptr) return 0;
			if (T->data.key == e) return 1;
			return FindChild(T->lchild, e) || FindChild(T->rchild, e);
		}
		
		
		void InvertTree(BiTree &T){
			if (T == nullptr) return;
			BiTree tmp = T->lchild;
			T->lchild = T->rchild;
			T->rchild = tmp;
			InvertTree(T->lchild);
			InvertTree(T->rchild);
		}
	\end{lstlisting}
	Manager.h
	\begin{lstlisting}
		#include "def.h"
		#include "SingleTree.h"
		
		class Manager {
			private:
			
			public:
			TU member[MaxSize];
			int length;
			int size;
			map<string, int> name_index;
			
			Manager();
			
			// 获取树名,返回索引
			int GetCommand1();
			
			// 添加一棵树
			status AddMember();
			
			// 删除一棵树
			status DelMember();
			
			// 显示当前结构
			void DispStructure();
			
		};
	\end{lstlisting}
	Manager.cpp
	\begin{lstlisting}
		#include "Manager.h"
		
		Manager::Manager() : length(0), size(MaxSize){
			
		}
		
		int Manager::GetCommand1(){
			string name;
			cout << "enter the name of the tree:" << endl;
			cin >> name;
			auto it = this->name_index.find(name);
			if (it == this->name_index.end()){
				cout << "can't find the tree!" << endl;
				return -1;
			}
			return it->second;
		}
		
		status Manager::AddMember() {
			if (this->length == size){
				return OVERFLOW;
			}
			string name;
			cout << "enter a name for this tree:" << endl;
			cin >> name;
			this->name_index.insert(pair<string, int>(name, this->length));
			this->member[this->length].name = name;
			TElemType definition[100];
			GetData(definition);
			
			int state = CreateBiTree(this->member[this->length].T, definition);
			if (state == ERROR) return ERROR;
			else{
				this->length++;
				return OK;
			}
		}
		
		status Manager::DelMember() {
			int index;
			index = this->GetCommand1();
			if (index != -1){
				int state =  ClearBiTree(this->member[index].T);
				if (state == OK) {
					return OK;
				}
				else return ERROR;
			}
			return ERROR;
		}
		
		void Manager::DispStructure() {
			cout << "---------------------" << endl;
			cout << "|-Manager" << endl;
			for (int i = 0; i < this->length; i++){
				if (this->member[i].T == nullptr) cout << "   |-null" << endl;
				else cout << "   |-" << this->member[i].name << endl;
			}
			cout << "---------------------" << endl;
		}
	\end{lstlisting}
\end{document}