\documentclass[supercite]{Experimental_Report}

\usepackage{algorithm, multirow}
\usepackage{algpseudocode}
\usepackage{amsmath}
\usepackage{amsthm}
\usepackage{framed}
\usepackage{mathtools}
\usepackage{subcaption}
\usepackage{xltxtra} %提供了针对XeTeX的改进并且加入了XeTeX的LOGO, 自动调用xunicode
%宏包(提供Unicode字符宏)
\usepackage{bm}
\usepackage{tikz}
\usepackage{tikzscale}
\usepackage{pgfplots}
\usepackage{listings}
\usepackage{xcolor}
\usepackage{fontspec}

%%%%%%settings%%%%%%%%%
\pgfplotsset{compat=1.16}
\setmonofont{Consolas}

\definecolor{mygreen}{rgb}{0,0.6,0}
\definecolor{mygray}{rgb}{0.5,0.5,0.5}
\definecolor{mymauve}{rgb}{0.58,0,0.82}
\lstset{ %
	backgroundcolor=\color{white},   % choose the background color
	basicstyle=\footnotesize\ttfamily,        % size of fonts used for the code
	columns=fullflexible,
	breaklines=true,                 % automatic line breaking only atwhitespace
	captionpos=b,                    % sets the caption-position to bottom
	tabsize=4,
	commentstyle=\color{mygreen},    % comment style
	escapeinside={\%*}{*)},          % if you want to add LaTeX within your code
	keywordstyle=\color{blue},       % keyword style
	stringstyle=\color{mymauve}\ttfamily,     % string literal style
	frame=single,
	rulesepcolor=\color{red!20!green!20!blue!20},
	% identifierstyle=\color{red},
	language=c++,
}

\begin{document}
	\section{附录D 基于邻接表图实现的源程序}
	\begin{center}
		\textbf{/* Gragh on LinkTable Sturcture */}
	\end{center}
	main.cpp
	\begin{lstlisting}
		#include "def.h"
		#include "SingleGraph.h"
		#include "Manager.h"
		
		int main() {
			int state;
			int op;
			int index;
			int key, key1, key2;
			char filename[30];
			VertexType v;
			Manager M;
			M.MenuDisp();
			cout << "enter your command:" << endl;
			cin >> op;
			while (op){
				switch(op){
					case 1:  // 创建新表
					state = M.NewGraph();
					if (state == OK) {
						cout << "create successfully. The structure is:" << 
						endl;
						M.DispStruc();
					}
					else cout << "wrong data set!" << endl;
					break;
					case 2:  // 删除一张图
					if ((index = M.GetCommand1()) != -1){
						M.DelGraph(index);
						cout << "delete successfully! The structure is:" << 
						endl;
						M.DispStruc();
						break;
					}
					case 3:
					M.DispStruc();
					break;
					case 4: // 创建图
					if ((index = M.GetCommand2()) != -1){
						VertexType V[MAX_VERTEX_NUM];
						KeyType VR[100][2];
						ReadData(V, VR);
						state = CreateGraph(M.elem[index], V, VR);
						M.IfDataSetError(state);
					}
					break;
					case 5:  // 销毁图
					if ((index = M.GetCommand1()) != -1){
						DestroyGraph(M.elem[index]);
						cout << "successfully." << endl;
					}
					break;
					case 6:  // 定位节点
					if ((index = M.GetCommand1()) != -1){
						cout << "enter the key:" << endl;
						cin >> key;
						state = LocateVex(M.elem[index], key);
						if (state == -1){
							cout << "can't find vertex!" << endl;
						}
						else{
							cout << "the index and name is: " << state << "," \
							<< M.elem[index].vertices[state].data.others << 
							endl;
						}
					}
					break;
					case 7:  // 修改节点值
					if ((index = M.GetCommand1()) != -1){
						M.GetKey(key);
						M.GetVertexValue(v);
						state = PutVex(M.elem[index], key, v);
						M.IfDataSetError(state);
					}
					break;
					case 8:  // 获取第一个邻接点
					if ((index = M.GetCommand1()) != -1){
						M.GetKey(key);
						state = FirstAdjVex(M.elem[index], key);
						if (state == -1){
							cout << "without firstadjvex!" << endl;
						}
						else{
							cout << "the first adjvex is: " << 
							M.elem[index].vertices[state].data.key << "," \
							<< M.elem[index].vertices[state].data.others << 
							endl;
						}
					}
					break;
					case 9:  // 返回v的邻接点w的下一节点的位序
					if ((index = M.GetCommand1()) != -1){
						cout << "node v: ";
						M.GetKey(key1);
						cout << "node w: ";
						M.GetKey(key2);
						state = NextAdjVex(M.elem[index], key1, key2);
						if (state == -1){
							cout << "without firstadjvex!" << endl;
						}
						else{
							cout << "the next adjvex is:" << 
							M.elem[index].vertices[state].data.key << "," \
							<< M.elem[index].vertices[state].data.others << 
							endl;
						}
					}
					break;
					case 10:  // 插入顶点
					if ((index = M.GetCommand1()) != -1){
						M.GetVertexValue(v);
						state = InsertVex(M.elem[index], v);
						M.IfDataSetError(state);
					}
					break;
					case 11:  // 删除节点
					if ((index = M.GetCommand1()) != -1){
						M.GetKey(key);
						state = DeleteVex(M.elem[index], key);
						M.IfDataSetError(state);
					}
					break;
					case 12:  // 插入弧
					if ((index = M.GetCommand1()) != -1){
						cout << "node v: ";
						M.GetKey(key1);
						cout << "node w: ";
						M.GetKey(key2);
						state = InsertArc(M.elem[index], key1, key2);
						M.IfDataSetError(state);
					}
					break;
					case 13:  // 删除弧
					if ((index = M.GetCommand1()) != -1){
						cout << "node v: ";
						M.GetKey(key1);
						cout << "node w: ";
						M.GetKey(key2);
						state = DeleteArc(M.elem[index], key1, key2);
						M.IfDataSetError(state);
					}
					break;
					case 14:  // DFS
					if ((index = M.GetCommand1()) != -1){
						DFSTraverse(M.elem[index], visit);
					}
					break;
					case 15:  // BFS
					if ((index = M.GetCommand1()) != -1){
						BFSTraverse(M.elem[index], visit);
					}
					break;
					case 16:  // 保存文件
					if ((index = M.GetCommand1()) != -1){
						cout << "enter the filename:" << endl;
						cin >> filename;
						state = SaveGraph(M.elem[index], filename);
						M.IfIoError(state);
					}
					break;
					case 17:  // 加载文件
					if ((index = M.GetCommand2()) != -1){
						cout << "enter the filename:" << endl;
						cin >> filename;
						state = LoadGraph(M.elem[index], filename);
						M.IfIoError(state);
					}
					break;
					case 18:  // 距离小于k的顶点集合
					if ((index = M.GetCommand1()) != -1){
						M.GetKey(key);
						cout << "enter k:" <<endl;
						int k;
						cin >> k;
						vector<int> LessThanK = 
						VerticesSetLessThanK(M.elem[index], key, k);
						cout << "vertices are:" << endl;
						for (int i : LessThanK){
							cout << M.elem[index].vertices[i].data.key << "," \
							<< M.elem[index].vertices[i].data.others << "  ";
						}
					}
					break;
					case 19:  // 最短距离
					if ((index = M.GetCommand1()) != -1){
						cout << "enter v:";
						M.GetKey(key1);
						cout << "enter w:";
						M.GetKey(key2);
						cout << "dist is: " << 
						ShortestPathLength(M.elem[index], key1, key2);
					}
					break;
					case 20:  // 连通分量数
					if ((index = M.GetCommand1()) != -1){
						cout << "the num is:" << 
						ConnectedComponentsNums(M.elem[index]);
					}
					break;
					default:
					cout << "wrong command!" << endl;
					break;
					
				}
				cout << endl;
				cout << endl;
				M.MenuDisp();
				cout << "enter your command:" << endl;
				cin >> op;
				
			}
			cout << "bye." << endl;
			
			return 0;
		}
	\end{lstlisting}
	def.h
	\begin{lstlisting}
		#pragma once
		#include <cstdio>
		#include <cstdlib>
		#include <cstring>
		#include <iostream>
		#include <string>
		#include <map>
		#include <algorithm>
		#include <vector>
		#include <queue>
		using namespace std;
		
		#define TRUE 1
		#define FALSE 0
		#define OK 1
		#define ERROR 0
		#define INFEASIBLE -1
		#define OVERFLOW -2
		#define MAX_VERTEX_NUM 20
		typedef int status;
		typedef int KeyType;
		typedef enum {DG,DN,UDG,UDN} GraphKind;
		typedef struct {
			KeyType  key;
			char others[20];
		} VertexType; //顶点类型定义
		
		
		typedef struct ArcNode {         //表结点类型定义
			int adjvex;              //顶点位置编号
			struct ArcNode  *nextarc;	   //下一个表结点指针
			ArcNode() : nextarc(nullptr){}
		} ArcNode;
		typedef struct VNode{				//头结点及其数组类型定义
			VertexType data;       	//顶点信息
			ArcNode *firstarc;      	 //指向第一条弧
			VNode() : firstarc(nullptr){}
		} VNode,AdjList[MAX_VERTEX_NUM];
		typedef struct LGragh{  //邻接表的类型定义
			string name;
			AdjList vertices;     	 //头结点数组
			int vexnum,arcnum;   	  //顶点数、弧数
			GraphKind  kind;        //图的类型
			LGragh() : vexnum(0), arcnum(0), kind(UDG), name("unnamed"){}
		} ALGraph;
	\end{lstlisting}
	SingleGraph.h
	\begin{lstlisting}
		#pragma once
		#include "def.h"
		
		// 读取数据
		void ReadData(VertexType V[],KeyType VR[][2]);
		
		// 创建图
		status CreateGraph(ALGraph &G,VertexType V[],KeyType VR[][2]);
		int find_index(const ALGraph &G, int key);  // 
		查找key对应顶点的下标-------------通用函数
		status AddVertex(ALGraph &G, int key1, int key2);  // 添加边
		
		// 销毁图
		status DestroyGraph(ALGraph &G);
		
		// 根据u在图G中查找顶点,查找成功返回位序,否则返回-1;
		int LocateVex(const ALGraph &G,KeyType u);
		
		//根据u在图G中查找顶点,查找成功将该顶点值修改成value,返回OK;
		//如果查找失败或关键字不唯一,返回ERROR
		status PutVex(ALGraph &G,KeyType u,VertexType value);
		
		//根据u在图G中查找顶点,查找成功返回顶点u的第一邻接顶点位序,否则返回-1
		int FirstAdjVex(const ALGraph &G,KeyType u);
		
		//根据u在图G中查找顶点,查找成功返回顶点v的邻接顶点相对于w的下一邻接顶点的位
		序,查找失败返回-1
		int NextAdjVex(const ALGraph &G,KeyType v,KeyType w);
		
		//在图G中插入顶点v,成功返回OK,否则返回ERROR
		status InsertVex(ALGraph &G,VertexType v);
		
		//在图G中删除关键字v对应的顶点以及相关的弧,成功返回OK,否则返回ERROR
		status DeleteVex(ALGraph &G,KeyType v);
		int GetIndex(const ALGraph &G, KeyType v);  // 
		和find_index用法相同-------------通用
		
		//在图G中增加弧<v,w>,成功返回OK,否则返回ERROR
		status InsertArc(ALGraph &G,KeyType v,KeyType w);
		int IfConflict(const ALGraph &G, KeyType v, KeyType w);  // 防止同一条弧
		被插两次
		
		//在图G中删除弧<v,w>,成功返回OK,否则返回ERROR
		status DeleteArc(ALGraph &G,KeyType v,KeyType w);
		
		// dfs
		status DFSTraverse(ALGraph &G,void (*visit)(VertexType));
		void dfs(const ALGraph &G, int index, void (*visit)(VertexType), int 
		ifvisit[]);
		
		//bfs
		status BFSTraverse(ALGraph &G,void (*visit)(VertexType));
		void visit(VertexType v);  //------------------------------------------
		通用
		
		//保存文件
		status SaveGraph(ALGraph G, char FileName[]);
		
		//加载文件
		status LoadGraph(ALGraph &G, char FileName[]);
		
		//与v距离小于k的顶点
		vector<int> VerticesSetLessThanK(const ALGraph &G, KeyType v,int k);
		vector<int> GetDist(const ALGraph &G, KeyType v);  // 辅助函数:返回各顶
		点和v的距离----------通用
		
		// v和w之间的最短距离
		int ShortestPathLength(const ALGraph &G, KeyType v, KeyType w);
		
		// 返回连通分量个数
		int ConnectedComponentsNums(const ALGraph &G);
		void dfs0(const ALGraph &G, int index, vector<int> &ifvisit);  // 辅助函
		数
	\end{lstlisting}
	SingleGraph.cpp
	\begin{lstlisting}
		#include "def.h"
		#include "SingleGraph.h"
		
		void ReadData(VertexType V[],KeyType VR[][2]){
			// 获取V数组元素
			int key;
			char others[30];
			int i = 0;
			cout << "enter the key-value pair end with \"-1 nil\"" << endl;
			cin >> key >> others;
			while (key != -1){
				V[i].key = key;
				strcpy(V[i].others, others);
				i++;
				cin >> key >> others;
			}
			V[i].key = key;
			strcpy(V[i].others, others);
			cout << "vertex read successfully. Enter the edge end with \"-1 
			-1\"" << endl;
			// 获取VR数组元素
			int key1, key2;
			i = 0;
			cin >> key1 >> key2;
			while (key1 != -1){
				VR[i][0] = key1;
				VR[i][1] = key2;
				i++;
				cin >> key1 >> key2;
			}
			VR[i][0] = key1;
			VR[i][1] = key2;
			cout << "read successfully!" << endl;
		}
		
		
		status CreateGraph(ALGraph &G,VertexType V[],KeyType VR[][2])
		{
			//判断keyの唯一性
			int hash[100] = {0};
			int i = 0;
			while (V[i].key != -1){
				if (hash[V[i].key] == 1){
					return ERROR;
				}
				hash[V[i].key] = 1;
				i++;
			}
			//图的操作
			// 添加顶点
			i = 0;
			G.vexnum = 0;
			G.arcnum = 0;
			G.kind = UDG;
			while (i < 20 && V[i].key != -1){
				G.vertices[i].data = V[i];
				G.vertices[i].firstarc = nullptr;
				G.vexnum++;
				i++;
			}
			if (i == 20 && V[i].key != -1) return ERROR; // 加入的顶点数不能溢出
			// 添加边
			i = 0;
			while (VR[i][0] != -1){
				int state = AddVertex(G, VR[i][0], VR[i][1]);
				if (state == ERROR) return ERROR;
				i++;
			}
			return OK;
		}
		
		status AddVertex(ALGraph &G, int key1, int key2) {
			int index1 = find_index(G, key1);
			if (index1 == -1) return ERROR;
			int index2 = find_index(G, key2);
			if (index2 == -1) return ERROR;
			
			VNode &Head1 = G.vertices[index1];
			ArcNode *NewNode1 = (ArcNode *) malloc(sizeof(struct ArcNode));
			NewNode1->adjvex = index2;
			NewNode1->nextarc = Head1.firstarc;
			Head1.firstarc = NewNode1;
			
			VNode &Head2 = G.vertices[index2];
			ArcNode *NewNode2 = (ArcNode *) malloc(sizeof(struct ArcNode));
			NewNode2->adjvex = index1;
			NewNode2->nextarc = Head2.firstarc;
			Head2.firstarc = NewNode2;
			
			G.arcnum++;
			return OK;
		}
		
		int find_index(const ALGraph &G, int key){
			for (int i = 0; i < G.vexnum; i++){
				if (G.vertices[i].data.key == key) return i;
			}
			return -1;
		}
		
		
		status DestroyGraph(ALGraph &G){
			for (int i = 0; i < G.vexnum; i++){
				ArcNode* p = G.vertices[i].firstarc;
				ArcNode* pre;
				while (p){
					pre = p;
					p = p->nextarc;
					free(pre);
				}
				G.vertices[i].firstarc = nullptr;
			}
			G.vexnum = 0;
			G.arcnum = 0;
			return OK;
		}
		
		
		int LocateVex(const ALGraph &G,KeyType u)
		{
			for (int i = 0; i < G.vexnum; i++){
				if (G.vertices[i].data.key == u){
					return i;
				}
			}
			return -1;
		}
		
		
		status PutVex(ALGraph &G,KeyType u,VertexType value)
		{
			int hash[100] = {0};
			int i;
			for (i = 0; i < G.vexnum; i++){
				hash[G.vertices[i].data.key] = 1;
			}
			if (hash[u] == 0 || u != value.key && hash[value.key] == 1) return 
			ERROR;
			for (i = 0; i < G.vexnum; i++){
				if (u == G.vertices[i].data.key){
					G.vertices[i].data = value;
				}
			}
			return OK;
		}
		
		
		int FirstAdjVex(const ALGraph &G,KeyType u)
		{
			int i;
			for (i = 0; i < G.vexnum; i++){
				if (G.vertices[i].data.key == u) break;
			}
			if (i == G.vexnum) return -1;
			if (G.vertices[i].firstarc->adjvex) return 
			G.vertices[i].firstarc->adjvex;
			else return -1;
		}
		
		
		int NextAdjVex(const ALGraph &G,KeyType v,KeyType w)
		{
			int i;
			for (i = 0; i < G.vexnum; i++){
				if (G.vertices[i].data.key == v) break;
			}
			if (i == G.vexnum) return -1;
			ArcNode* p = G.vertices[i].firstarc;
			while (p){
				if (G.vertices[p->adjvex].data.key == w){
					if (p->nextarc) return p->nextarc->adjvex;
					else return -1;
				}
				p = p->nextarc;
			}
			return -1;
		}
		
		
		status InsertVex(ALGraph &G,VertexType v)
		{
			int hash[100] = {0};
			int i;
			for (i = 0; i < G.vexnum; i++){
				hash[G.vertices[i].data.key] = 1;
			}
			if (hash[v.key] == 1) return ERROR;
			if (G.vexnum == 20) return ERROR;
			G.vertices[G.vexnum++].data = v;
			G.vertices[G.vexnum].firstarc = nullptr;
			return OK;
		}
		
		
		status DeleteVex(ALGraph &G,KeyType v)
		{
			int index = GetIndex(G, v);
			if (index == -1) return ERROR;
			if (G.vexnum == 1) return ERROR;
			// 删除节点
			ArcNode* p = G.vertices[index].firstarc;
			// 防止p本来就是null
			if (p != nullptr){
				ArcNode* next = p->nextarc;
				while (next){
					free(p);
					p = next;
					next = next->nextarc;
				}
				free(p);
				G.vertices[index].firstarc = nullptr;
			}
			for (int i = index; i < G.vexnum-1; i++){
				G.vertices[i] = G.vertices[i+1];
			}
			G.vexnum--;
			// 删除边
			for (int i = 0; i < G.vexnum; i++){
				p = G.vertices[i].firstarc;
				// 这里分类稍微麻烦,要看看第一个节点是否就是被删的节点
				if (p && p->adjvex == index) {
					G.vertices[i].firstarc = p->nextarc;
					G.arcnum--;
					free(p);
				}
				else{
					while (p && p->nextarc){
						if (p->nextarc->adjvex == index){
							ArcNode* bin = p->nextarc;
							p->nextarc = p->nextarc->nextarc;
							free(bin);
							G.arcnum--;
							break;
						}
						p = p->nextarc;
					}
				}
			}
			// 调整删除节点后adjvex的值
			for (int i = 0; i < G.vexnum; i++){
				p = G.vertices[i].firstarc;
				while(p){
					if (p->adjvex > index) p->adjvex--;
					p = p->nextarc;
				}
			}
			return OK;
		}
		
		int GetIndex(const ALGraph &G, KeyType v) {
			for (int i = 0; i < G.vexnum; i++) {
				if (G.vertices[i].data.key == v) return i;
			}
			return -1;
		}
		
		
		status InsertArc(ALGraph &G,KeyType v,KeyType w)
		{
			int index1 = GetIndex(G, v);
			int index2 = GetIndex(G, w);
			if (index1 == -1 || index2 == -1) return ERROR;
			if (IfConflict(G, v, w)) return ERROR;
			
			ArcNode* p = G.vertices[index1].firstarc;
			ArcNode* newnode = (ArcNode*)malloc(sizeof(struct ArcNode));
			newnode->adjvex = index2;
			newnode->nextarc = p;
			G.vertices[index1].firstarc = newnode;
			G.arcnum++;
			
			ArcNode* p0 = G.vertices[index2].firstarc;
			ArcNode* newnode1 = (ArcNode*)malloc(sizeof(struct ArcNode));
			newnode1->adjvex = index1;
			newnode1->nextarc = p0;
			G.vertices[index2].firstarc = newnode1;
			
			return OK;
		}
		
		int IfConflict(const ALGraph &G, KeyType v, KeyType w){
			for (int i = 0; i < G.vexnum; i++){
				if (G.vertices[i].data.key == v){
					ArcNode* p = G.vertices[i].firstarc;
					int target = GetIndex(G, w);
					while (p){
						if (p->adjvex == target) return 1;
						p = p->nextarc;
					}
				}
			}
			return 0;
		}
		
		
		status DeleteArc(ALGraph &G,KeyType v,KeyType w)
		{
			int index1 = GetIndex(G, v);
			int index2 = GetIndex(G, w);
			if (index1 == -1 || index2 == -1) return ERROR;
			int flag = 0;
			// 删弧
			ArcNode* p1 = G.vertices[index1].firstarc;
			if (p1 && p1->adjvex == index2) {
				flag = 1;
				G.vertices[index1].firstarc = p1->nextarc;
				G.arcnum--;  // 弧计数在这里减去就行,下面不用
				free(p1);
			}
			else{
				while (p1->nextarc){
					if (p1->nextarc->adjvex == index2){
						flag = 1;  // 同样的,这里找到弧设置一下就行,下面不用
						ArcNode* bin = p1->nextarc;
						p1->nextarc = p1->nextarc->nextarc;
						free(bin);
						G.arcnum--;
						break;
					}
					p1 = p1->nextarc;
				}
			}
			// 对称情况
			ArcNode* p2 = G.vertices[index2].firstarc;
			if (p2 && p2->adjvex == index1) {
				G.vertices[index2].firstarc = p2->nextarc;
				//G.arcnum--;
				free(p2);
			}
			else{
				while (p2->nextarc){
					if (p2->nextarc->adjvex == index1){
						ArcNode* bin = p2->nextarc;
						p2->nextarc = p2->nextarc->nextarc;
						free(bin);
						//G.arcnum--;
						break;
					}
					p2 = p2->nextarc;
				}
			}
			
			if (flag == 0) return ERROR;
			return OK;
		}
		
		
		status DFSTraverse(ALGraph &G,void (*visit)(VertexType))
		{
			int ifvisit[20] = {0};
			for (int i = 0; i < G.vexnum; i++){
				dfs(G, i, visit, ifvisit);
			}
			return OK;
		}
		
		void dfs(const ALGraph &G, int index, void (*visit)(VertexType), int 
		ifvisit[]){
			if (ifvisit[index] == 1) return;
			visit(G.vertices[index].data);
			ifvisit[index] = 1;
			ArcNode* p = G.vertices[index].firstarc;
			while (p){
				dfs(G, p->adjvex, visit, ifvisit);
				p = p->nextarc;
			}
		}
		
		
		status BFSTraverse(ALGraph &G,void (*visit)(VertexType))
		{
			int q[100];
			int ifvisit[100];
			for (int i = 0; i < G.vexnum; i++){
				ifvisit[i] = 0;
			}
			int head = 0, tail = 0;
			for (int i = 0; i < G.vexnum; i++){
				if (ifvisit[i] == 0){
					q[tail++] = i;
					ifvisit[i] = 1;
				}
				while (head != tail){
					VNode cur = G.vertices[q[head++]];
					visit(cur.data);
					ArcNode* p = cur.firstarc;
					while (p){
						if (ifvisit[p->adjvex] == 0){
							q[tail++] = p->adjvex;
							ifvisit[p->adjvex] = 1;
						}
						p = p->nextarc;
					}
				}
			}
			return OK;
		}
		
		void visit(VertexType v)
		{
			printf(" %d %s",v.key,v.others);
		}
		
		
		status SaveGraph(ALGraph G, char FileName[])
		{
			FILE* fp = fopen(FileName, "w");
			if (!fp) return ERROR;
			for (int i = 0; i < G.vexnum; i++){
				fprintf(fp, "%d %s\n", G.vertices[i].data.key, 
				%%G.vertices[i].data.others);
			}
			char last[30] = "nil";
			fprintf(fp, "%d %s\n", -1, last);
			for (int i = 0; i < G.vexnum; i++){
				ArcNode* p = G.vertices[i].firstarc;
				while (p){
					fprintf(fp, "%d %d\n", G.vertices[i].data.key, 
					%%G.vertices[p->adjvex].data.key);
					p = p->nextarc;
				}
			}
			fprintf(fp, "%d %d\n", -1, -1);
			fclose(fp);
			return OK;
		}
		
		
		status LoadGraph(ALGraph &G, char FileName[])
		{
			FILE* fp = fopen(FileName, "r");
			if (!fp) return ERROR;
			int key;
			char others[30];
			fscanf(fp, "%d %s", &key, others);
			while (key != -1){
				G.vertices[G.vexnum].data.key = key;
				strcpy(G.vertices[G.vexnum].data.others, others);
				G.vexnum++;
				fscanf(fp, "%d %s", &key, others);
			}
			KeyType key1, key2;
			fscanf(fp, "%d %d", &key1, &key2);
			while (key1 != -1){
				InsertArc(G, key1, key2);
				fscanf(fp, "%d %d", &key1, &key2);
			}
			fclose(fp);
			return OK;
		}
		
		
		vector<int> VerticesSetLessThanK(const ALGraph &G, KeyType v, int k){
			vector<int> dist = GetDist(G, v);
			// 统计距离小于k的节点
			vector<int> LessThanK;
			for (int i = 0; i < G.vexnum; i++){
				if (dist[i] < k){
					LessThanK.push_back(i);
				}
			}
			return LessThanK;
		}
		
		vector<int> GetDist(const ALGraph &G, KeyType v){
			vector<int> ifvisited(G.vexnum, 0);
			vector<int> dist(G.vexnum, INT16_MAX);
			queue<int> q;
			int cur_size;
			int index;
			int cur_dist = 0;
			//bfs过程中记录距离
			int index0 = GetIndex(G, v);
			dist[index0] = 0;
			ifvisited[index0] = 0;
			q.push(index0);
			cur_size = q.size();
			while (!q.empty()){
				for (int i = 0; i < cur_size; i++){
					index = q.front();
					q.pop();
					dist[index] = cur_dist;
					// 加入与该点邻接的点
					ArcNode *p = G.vertices[index].firstarc;
					while (p){
						if (ifvisited[p->adjvex] == 0){
							q.push(p->adjvex);
							ifvisited[p->adjvex] = 1;
						}
						p = p->nextarc;
					}
				}
				cur_dist++;
				cur_size = q.size();
			}
			return dist;
		}
		
		
		int ShortestPathLength(const ALGraph &G, KeyType v, KeyType w){
			vector<int> dist = GetDist(G, v);
			return dist[GetIndex(G, w)];
		}
		
		
		int ConnectedComponentsNums(const ALGraph &G){
			vector<int> ifvisited(G.vexnum, 0);
			int num = 0;
			for (int i = 0; i < G.vexnum; i++){
				if (ifvisited[i] == 0){
					num++;
					dfs0(G, i, ifvisited);
				}
			}
			return num;
		}
		
		void dfs0(const ALGraph &G, int index, vector<int> &ifvisit){
			if (ifvisit[index] == 1) return;
			ifvisit[index] = 1;
			ArcNode* p = G.vertices[index].firstarc;
			while (p){
				dfs0(G, p->adjvex, ifvisit);
				p = p->nextarc;
			}
		}
	\end{lstlisting}
	Manager.h
	\begin{lstlisting}
		#include "def.h"
		#include "SingleGraph.h"
		#pragma once
		
		class Manager {
			private:
			
			public:
			ALGraph elem[10];
			int length;
			int initial_size;
			map<string, int> name_index;  // 图名索引
			
			Manager();
			
			//展示菜单
			void MenuDisp();
			
			// 交互判空函数,为空报错
			int GetCommand1();
			
			// 判不空交互函数,不为空报错
			int GetCommand2();
			
			// 数据集错误检查
			void IfDataSetError(int state);
			
			// 判断io错误
			void IfIoError(int state);
			
			// 获取key
			void GetKey(int &key);
			
			// 获取一个节点值
			void GetVertexValue(VertexType &v);
			
			//    // 定位节点位序检查,找不到报错
			//    void IfFind(int state);
			
			//创建一张图
			status NewGraph();
			
			//删除一张图
			status DelGraph(int index);
			
			//展示树形结构
			void DispStruc();
		};
	\end{lstlisting}
	Manager.cpp
	\begin{lstlisting}
		#include "Manager.h"
		
		Manager::Manager() : length(0), initial_size(10){}
		
		void Manager::MenuDisp() {
			cout << "                  Menu" << endl;
			cout << "--------------------------------------------" << endl;
			cout << "  1. NewGraph                 2. DelGraph" << endl;
			cout << "  3. DispStruc                           " << endl;
			cout << "  4. CreateGraph              5. DestroyGraph" << endl;
			cout << "  6. LocateVex                7. PutVex" << endl;
			cout << "  8. FirstAdjVex              9. NextAdjVex" << endl;
			cout << "  10. InsertVex               11. DeleteVex" << endl;
			cout << "  12. InsertArc               13. DeleteArc" << endl;
			cout << "  14. DFS                     15. BFS" << endl;
			cout << "  16. Save                    17. Load" << endl;
			cout << "  18. VerticesSetLessThanK    19. ShortestPathLength" << 
			endl;
			cout << "  20. ConnectedComponentsNums  0. quit" << endl;
			cout << "--------------------------------------------" << endl;
		}
		
		
		int Manager::GetCommand1(){
			string name;
			cout << "enter the name of the graph to be operated:" << endl;
			cin >> name;
			map<string, int>::iterator it = this->name_index.find(name);
			if (it == this->name_index.end()) {
				cout << "can't find the graph!" << endl;
				return -1;
			}
			else if (this->elem[it->second].vexnum == 0){
				cout << "empty graph!" << endl;
				return -1;
			}
			else return it->second;
		}
		
		int Manager::GetCommand2(){
			string name;
			cout << "enter the name:" << endl;
			cin >> name;
			map<string, int>::iterator it = this->name_index.find(name);
			if (it == this->name_index.end()) {
				cout << "can't find the graph!" << endl;
				return -1;
			}
			else if (this->elem[it->second].vexnum != 0){
				cout << "existed graph!" << endl;
				return -1;
			}
			else return it->second;
		}
		
		void Manager::IfDataSetError(int state) {
			if (state == ERROR){
				cout << "data set error!" << endl;
			}
			else{
				cout << "successfully." << endl;
			}
		}
		
		void Manager::IfIoError(int state) {
			if (state == ERROR){
				cout << "IO ERROR!" << endl;
			}
			else{
				cout << "successfully." << endl;
			}
		}
		
		void Manager::GetKey(int &key) {
			cout << "enter the key:" << endl;
			cin >> key;
		}
		
		void Manager::GetVertexValue(VertexType &v) {
			cout << "enter the key-value pair:" << endl;
			cin >> v.key >> v.others;
		}
		//void Manager::IfFind(int state) {
			//    if (state == -1){
				//        cout << "can't find vertex!" << endl;
				//    }
			//    else{
				//        cout << ""
				//    }
			//}
		
		status Manager::NewGraph() {
			VertexType V[MAX_VERTEX_NUM];
			KeyType VR[100][2];
			cout << "enter a name for this graph:" << endl;
			cin >> this->elem[this->length].name;
			// 读取数据
			ReadData(V, VR);
			// 创建图
			int state = CreateGraph(this->elem[this->length], V, VR);
			if (state == OK) {
				this->name_index.insert(pair<string 
				,int>(this->elem[this->length].name, this->length));
				this->length++;
			}
			return state;
		}
		
		status Manager::DelGraph(int index) {
			return DestroyGraph(this->elem[index]);
		}
		
		void Manager::DispStruc() {
			cout << "|-Manager" << endl;
			for (int i = 0; i < this->length; i++){
				if (this->elem[i].vexnum == 0){
					cout << "    |-null" << endl;
				}
				else {
					cout << "    |-" << this->elem[i].name << endl;
				}
			}
		}
	\end{lstlisting}
\end{document}
